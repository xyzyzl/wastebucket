\documentclass{scrartcl}
\usepackage[utf8]{inputenc}
\usepackage[sexy]{evan}

\title{Euler Circle Chapter 7: Constructability + Galois Fields}
\author{Albert Ye}
\date{\today}

\begin{document}
\maketitle
\subection{Notes}
\subsubsection{Construction}
The ancient Greeks wondered what shapes they could/couldn't construct with just a straightedge.

Could construct segments of length $a+b$, $|a-b|$, $ab$, $a/b$, $\sqrt{a}$.

Three things they couldn't do:
\begin{itemize}
	\item \textbf{Trisect angle.} Given \textit{any} angle, trisect.
	\item \textbf{Duplicate cube.} Given some cube, find the cube with twice the area.
	\item \textbf{Square circle.} Given some circle, find the square that bisects it.
\end{itemize}

Ground rules: We have a segment of length $1$ with endpoints $(0,0)$ and $(1,0)$. We can
\begin{itemize}
	\item Draw a line between two constructed points
	\item Find the point of intersection of two non-parallel lines.
	\item Construct a circle with center $P$ and radius $a$
	\item Find points of indersection between line/circle and circle/circle. (if they exist)
\end{itemize}
Here's how to multiply lengths with compass/straightedge: Take lines of length $a, b$. Find the point where they intersect, and take the point on one line that's one unit away from intersection. Then Just similar-triangles away.

Divide follows trivially.

\begin{definition}
	We call the collection of lengths of constructible segments, as well as their negatives, the \vocab{constructable} numbers.
\end{definition}

The constructable numbers form a field as the sum, product, quotient, and difference of constructable numbers is constructable. This field contains $\mathbb{Q}$, but also some irrationals. Note that the field of constructible numbers is contained in $\mathbB{R}$. It contains, for instance, $\sqrt{2}$. Call the field of constructible numbers $\mathbb{F}$. 

What does $\mathbb{F}$ contain? Note that we can consider a point $(x, y) \in \mathbb{R}^2$ to be constructible if $x,y \in \mathbb{F}$. Suppose $P, Q$ are constructible points. The line through them has the equation $ax+by+c=0$, where $a,b,c \in \mathbb{F}$.

Similarly, if $P$ has a constructible point and $r \in \mathbb{F}$, then the circle centered at $P$ with radius $r$ can be written in the form $(x-h)^2 + (y-k)^2 = r^2$, where $h,k,r \in \mathbb{F}$. The only way to generate \textit{new} points is by intersecting lines and circles.

\begin{itemize}
	\item \textbf{Intersection of lines.} If $a,b,c,d,e,f \in \mathbb{F}$, then any intsersection of the lines $ax+by+c = 0$ and $dx+ey+f = 0$ also has coordinates in $\mathbb{F}$. 
	\item \textbf{Intersection of line and circle.} If $a,b,c,h,k,r \in \mathbb{F}$, then any intersection of the line and the circle lies in a quadratic extension of $\mathbb{F}$, or an extension of the form $\mathbb{F}(\sqrt{x})$ for some positive $x \in \mathbb{F}$. 
	\item Intersection twocircles: If $h_1, k_1, r_1, h_2, k_2, r_2 \in \mathbb{F}$, then any intersection of the circles $(x-h_1)^2 + (y-k_1)^2 = r_1^2$ and $(x-h_2)^2 + (y-k_2)^2 = r_2^2$ lies in a quadratic extension of $\mathbb{F}$.
\end{itemize}

Thus we see that any individual step in the construction at most increases the degree of the field of numbers thus far constructed by a factor of $2$. Thus any field of constructible numbers must have degree $2^n$ for some $n$. On the other hand, an subfield of $\mathbb{R}$ that can be obtained by successively adjoining sqrts of positive elements is a field of constructible numbers, because we know how to take square roots. Thus, we have

\begin{theorem}
	A number $\alpha \in \mathbb{R}$ is constructible iff there is a sequence of fields $F_0 = \mathbb{Q} \subset F_1 \subset \cdots \subset F_n \subset \mathbb{R}$.
\end{theorem}

\subsubsection{Solving the Greek Problems}
\begin{problem}[Trisecting the angle]
	Prove that most angles can't be trisected.
\end{problem}

\begin{solution}
	We claim we \textit{cannot} trisect a $\dfrac{\pi}{3}$ angle. Suppose we can construct $\theta = \pi/9$ using a compass and straightedge, which we assume is measured wrt the positive $x$-axis. We can find a point of distance 1 from the origin and angle $\theta$, i.e. $(\cos \theta, \sin \theta)$. Thus $\cos \theta$ must be constructible. Apply the triple-angle identity for $\cos$:
	$$\cos 3\theta = 1 - \cos ^3 \theta - 3 \cos \theta.$$
	Letting $\theta = \frac{\pi}{9}$, we have $\cos 3\theta = \dfrac{1}{2}$. Hence $\cos \theta$ is a root of the cubic polynomial $4x^3 - 3x - \dfrac{1}{2}$ or $8x^3 - 6x - 1$. Let $y = 2x$, so we have $y^3 - 3y - 1 = 0$. This polynomial is irreducible. Therefore, $[\mathbb{Q}(2 \cos \frac{\pi}{9} : Q] = 3,$ and so $[\mathbb{Q}(\cos \frac{\pi}{9} : \mathbb{Q}] = 3$. Thus, by the Tower Law, $\cos \frac{\pi}{9}$ cannot lie in any constructible field, for its degree is not a power of 2. Thus, we can't construct a $\frac{\pi}{9}$ angle.
\end{solution}

\begin{problem}[Doubling the cube]
	Prove that we cannot construct a cube with side length $\cbrt 2$.
\end{problem}

\begin{solution}
	We wish to construct of segment of length $a = \cbrt{2}$. $a$ is a root of $x^3-2=0$, which is irreducible by Eisenstein. Hence, $[\mathbb{Q}(a) : \mathbb{Q}] = 3$, so $a$ can't be constructed.
\end{solution}

\begin{problem}[Squaring the circle]
	Prove that we can't construct a square with the same area as a given circle.
\end{problem}

\begin{solution}
	We take it as given that $\pi$ is transcendental, so $\sqrt \pi$ is also transcendental. A circle with radius $1$ has area $\pi$, so if we have a square of the same area, it must have sidelength $\sqrt{\pi}$. Since $\sqrt{\pi}$ is transcendental, it can't be constructed.
\end{solution}

\subsubsection{Splitting Fields}
Let $F$ be a field and $f(x)$ an irreducible polynomial with coefficients in $F$. Adjoining one or all roots my yield different fields.
\begin{definition}
	Let $F$ be a field and $f$ a nonzero polynomial with coefficients in $F$. We say that an extension $L/F$ is a \vocab{splitting field} for $f$ if all the roots of $f$ lie in $L$, but not in any smaller extension of $F$.
\end{definition}
Equivalently, the splitting field of $f$ is the field obtained by adjoining all roots of $f$ to $F$.
\begin{example}
	\begin{itemize}
		\item If $a \in Q$ is not a perfect square, then the splitting field of $x^2-a$ is $\mathbb{Q}(\sqrt{a})$, since the other root $-\sqrt{a}$ is already in this field. Its degree is $2$. 
		\item The splitting field of the polynomials $x^2+1$ and $x^2-2$ is $\mathbb{Q}(\sqrt{-1}, \sqrt{2})$. It has degree $4$ over $\mathbb{Q}$, and it contains three quadratic subfields: $\mathbb{Q}(\sqrt{-1}), \mathbb{Q}(\sqrt{2}), \mathbb{Q}(\sqrt{-2})$. This is an example of a biquadratic field. This is also the splitting field of $(x^2+1)(x^2-2)$.
		\item The splitting field of the polynomial $x^3 - 2$ over $\mathbb{Q}$ is the field $\mathbb{Q}(\cbrt{2}, e^{2\pi i/3}$.
	\end{itemize}
\end{example}

Let $p$ be a prime, and consider the splitting field $L$ of $f(x) = x^p-2$. $f$ is irreducibleby Eisenstein with $p = 2$. Its roots are $\sqrt[p]{2} \zeta_p^k$ for $0 \le k \le p-1$ and $\zeta_p = e^{2\pi i/p}$. Thus $L = \mathbb{Q}(\cbrt{2}, \zeta_p)$. We have $$[L:\mathbb{Q}] = [L : \mathbb{Q}(\sqrt[p]{2})] \cdots [(\mathbb{Q}(\sqrt[p]{2} : \mathbb{Q}],$$ which is a multiple of $p$. On the other hand, $$[L:Q] = [L:\mathbb{Q}(\zeta_p)] \cdot [\mathbb{Q}(\zeta_p) : \mathbb{Q}].$$ Now we have to compute $[\mathbb{Q}(\zeta_p : \mathbb{Q}]$. Note that $\zeta_p$ is a root of $x^p-1$ but isn't $1$, so it's a root of $x^{p-1} + x^{p-2} + \cdots + x + 1 = \Phi_p(x).$ Exercise: check that $\Phi_p(x)$ is irreducible. Thus $[\mathbb{Q}(\zeta_p) : \mathbb{Q}] = p-1$. Thus $[L : \mathbb{Q}]$ is a multiple of $p-1$, and so $[L : \mathbb{Q}]$ is a multiple of $p(p-1)$.

Suppose $K/F$ is a field extension and $\alpha$ is algebraic over $F$. How do we compare $F(\alpha) : F]$ to $[K(\alpha) : K]$? Since $[F(\alpha):F]$ is the degree of the minimal polynomial of $\alpha$ over $F$ and $[K(\alpha) : K]$ is the degree of the minimal polynomial over $K$, we must have $[K(\alpha) : K] \le [F(\alpha) : F]$.

Now, we have $[L : \mathbb{Q}(\zeta_p)] \le [\mathbb{Q}(\sqrt[p]{2}) : \mathbb{Q}]$. Thus we have $[L:\mathbb{Q} \le p(p-1)$, so it's equal.

\subsubsection{Algebraic Closures}
\begin{definition}
	A field $F$ is said to be \vocab{algebraically closed} if all irreducible polynomials over $F$ have degree $1$.
\end{definition}
For example, $\mathbb{C}$ is algebraically closed, and $\overline{\mathbb{Q}}$ (Q-bar) is too. Q-bar is equivalent to the set of algebraic numbers.

\subsubsection{Field Automorphisms and fixed fields}
$\Aut(K)$ is the set of automorphisms of $K$. In fact, it's a group. Suppose $K$ is a field and $\sigma \in \Aut(K)$. Then there are some $x \in K$ s.t. $\sigma(x) = x$. These elements are said to be \vocab{fixed} by $\sigma$.

\begin{proposition}
	If $\sigma \in \Aut(K)$, then the set of elements of $K$ fixed by $\sigma$ forms a field.
\end{proposition}

\begin{proof}
	Since $\sigma$ is a homomorphism, $\sigma(0) = 0$ and $\sigma(1) = 1$, so $0, 1$ are fixed. Suppose $a,b$ are fixed. Then 
	\begin{align*}
		\sigma(a+b) &= \sigma(a) + \sigma(b) = a + b \\
		\sigma(ab) &= \sigma(a) \sigma(b) = ab \\
		\sigma(-a) &= -\sigma(a) = -a \\
		\sigma(a^{-1} = \sigma(a)^{-1} = a^{-1}.
	\end{align*}
	Thus, the set of fixed is closed, so it is a field.
\end{proof}

\begin{proposition}
	\begin{itemize}
		\item If $L/K/F$ is a tower of field extensions, then $\Aut(L/K) \le \Aut(L/f)$.
		\item If $H_1 \le H_2 \le \Aut(K)$, then $K^{H_1} \subset K^{H_2}$.
	\end{itemize}
\end{proposition}

\begin{proposition}
	Let $K/F $ be a field extension. Let $\sigma \in \Aut(K/F)$, let $\alpha \in K$ be algebraic over $F$, and let $f(x)$ be the minimal polynomial for $\alpha$ over $F$. Then $\sigma(\alpha)$ is a root of $f(x)$. 
\end{proposition}

\begin{proof}
	Suppose $f(x) = x^n - a_{n-1} x^{n-1} + \cdots + a_1x + a_0$. Then, we have 
	\begin{align*}
		0 &= \sigma(f(\alpha)) \\
		&= \sigma(\alpha^n + a_{n-1}\alpha^{n-1} + \cdots + a_1 \alpha + a_0) \\
		&= \sigma(\alpha)^n + \cdots + \sigma(a_1\alpha) + \sigma(a_0) \\
		&= \sigma(\alpha)^n + a_{n-1} \sigma(\alpha)^{n-1} + \cdots + a_1 \sigma(\alpha) + a_0 \\
		&= f(\sigma(\alpha)).
	\end{align*}
	Thus, $\sigma(\alpha)$ is a root of $f$.
\end{proof}

Note also that an automorphism is entirely specified by its behavior on a generating set of the field. If $K = F(\alpha)$, for instance, then $\sigma \in \Aut(K/F)$ is entirely determined by the value of $\sigma(\alpha)$.

\begin{proposition}
	Let $\alpha$ be algebraic over $F$, and let $K = F(\alpha)$. Then $|\Aut(K/F)| \le [K:F]$.
\end{proposition}

In general, Proposition 5.4 is supposed to be equality in good cases. It isn't in the case of $\mathbb{Q}(\cbrt{2})$ because the other things aren't automorphisms, but only isomorphisms between different fields. This is because $\mathbb{Q}(\cbrt{2})$ isn't a splitting field. If $K$ is the splitting field of $f(x)$ over $F$, then $\Aut(K/F) = [K:F]$.

There's one more thing that can go wrong, which is that the minimal polynomial $f$ of $\alpha$ could have a double root, i.e there is some $\alpha \in K$ s.t. $(x-a)^2 | f(x)$. In this case, if $f(x)$ is irreducible, $[F(\alpha) : F] = \deg(f)$, but there aren't enough roots of $f$ to have $\deg(f)$ automorphisms.

This doesn't happen in fields we're familiar with like finite extensions of $\mathbb{Q}$. To see this, look at $f'(x)$, the derivative of $f$. If $(x-a)^2$ is a factor of $f(x)$, then $x-a$ is a factor of $f'(x)$, and $\deg(f') = \deg(f) - 1$. Also, if the coefficients of $f$ lie in $F$, then so do the coefficients of $f'$. Thus, a polynomial with multiple roots cannot be a minimal polynomial.

However, this can happen in other fields in $\mathbb{F}_p(t)$. We can still take derivatives here, and the derivative has pretty much the same properties as it does over $\mathbb{Q}$ or $\mathbb{R}$, but with one subtle point: the derivative could be $0$. For example, the polynomial $x^p-i$ over $\mathbb{F}_p(t)$ is the minimal polynomial of $\sqrt[p]{t}$, and it has only one root because $x^p-t = (x-\sqrt[p]{t})^p$ in $\mathbb{F}_p(t)$. Note that the derivative of $x^p-t$ is $0$.

\begin{definition}
	An algebraic field extension $K/F$ is said to be \vocab{separable} if, for any $\alpha \in K$, the minimal polynomial of $\alpha$ has all distinct roots.
\end{definition}

\subsubsection{Galois extensions/groups}
Galois extensions: nicest extensions, right number of automorphisms.
\begin{definition}
	An algebraic extension $K/F$ is said to be \vocab{Galois} if $K^{\Aut(K/F)} = F$.
\end{definition}

\begin{theorem}
	The following are equivalent for an algebraic extension $K/F$:
	\begin{enumerate}
		\item $K/F$ Galois
		\item $K/F$ is normal and separable.
		\item $|\Aut(K/F)| = [K:F]$
	\end{enumerate}
\end{theorem}

\begin{definition}
	If $K/F$ is Galois, we write $\Gal(K/F)$ instead of $\Aut(K/F)$. We call $\Gal(K/F)$ the \vocab{Galois group} of $K/F$.
\end{definition}
\end{document}
