\documentclass{article}
\usepackage[utf8]{inputenc}
\usepackage{amsmath}

\title{CP Problems Analysis}
\author{xyzyzl}
\date{4/6/2020}

\begin{document}
\maketitle

\section{Codeforces}
\textbf{1327D. Infinite Path}\\
Think of each permutation mapping as a graph. Let $a \rightarrow b$ be an edge if $P_a=b$. It's clear that the minimum value of $k$ s.t $C_{P^k_x} = C_x$ for some $x$ would be the smallest value where the node $kx$ away from the current node has the same color as the current node for all nonnegative integers $x$. As this is the minimum such $k$, it follows that $k | n$. Thus, we can loop over all divisors of $n$ and check if it satisfies the condition. \\
The mistake I made was that I did not remember to check visited for the \textit{current} node, but rather checked it on the \textit{next} node. Thus, the next iteration, it would see that the node was "visited" when it really was not, and break. Thus, all of my answers were just 1.

\section{USACO}
\textbf{2020 US Open - Gold: Haircut}\\
This was actually extremely simple and I don't know why I didn't get it.\\
When I found inversions with the algorithm, for each index $i$, it counted the number of elements $j < i$ that had $a_j < a_i$ using a BIT. Then, it subtracted that value from $i$. \\
So we can find the number inversions for all indices $i$ of a specific value. And that's all we really want. From there we can find the number of inversions solely comprising values $\le x \forall x$. \\
\textbf{Update}: I read my code for \textit{mincross} which had inversion count and I feel terrible for not noticing this -- it was literally in the code. \\
\\
\textbf{2020 US Open - Gold: Exercise}\\
So I made significant progress during the contest but for some reason just didn't implement it. A few key ideas that I came up with were that:
\begin{itemize}
	\item Each permutation is a "cycle" of different permutations. The number of times before repetition is the lcm of all cycle lengths. \\
	\item We need to find $lcm(a_1, a_2, \dotsc, a_g)$ for some $g$ s.t. $\sum_{i=1}^g a_i = n$. \\
	\item Anything that works for $j < i$ steps also works for $i$ steps.
\end{itemize}
However, I did not know how to find the lcm of all $a_i$ despite it being clear from the third observation that it was DP. \\
Let $dp_{i, j}$ be the running sum with the first $i$ primes and the first $j$ indices. Then
$$dp_{i,j} = dp{i-1, j} + (p_i^a \cdot dp_{i-1, j-p_i^a} \forall a \le \log_{p_i}j).$$
$dp_{i-1, j}$ is the previous step and the other long expression is the added running sum from the $i$th prime $p_i$.
\end{document}
